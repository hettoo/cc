\documentclass{article}

\usepackage[latin1]{inputenc}
\usepackage[english]{babel}
\usepackage{mathtools}
\usepackage{amsfonts}
\usepackage{amssymb}
\usepackage[nounderscore]{syntax}

\setlength{\grammarparsep}{0pt}
\renewcommand{\syntleft}{\normalfont\itshape}
\renewcommand{\syntright}{}

\title{SPL Parser Grammar}
\author{Tom Sanders and Gerco van Heerdt}

\begin{document}

\maketitle

First we recall the original grammar of the Simple Programming Language.
\setlength{\grammarindent}{6.8em}
\begin{grammar}
    <SPL> ::= <Decl>+

    <Decl> ::= <VarDecl> | <FunDecl>

    <VarDecl> ::= <Type> <id> `=' <Exp> `;'

    <FunDecl> ::= <RetType> <id> `(' [\,<FArgs>\,] `)' `{' <VarDecl>* <Stmt>+ `}'

    <RetType> ::= <Type> | `Void'

    <Type> ::= <id> | <BasicType> | `(' <Type> `,' <Type> `)' | `[' <Type> `]'

    <BasicType> ::= `Int' | `Bool' | `Char'

    <FArgs> ::= [\,<FArgs> `,'\,] <Type> <id>

    <Stmt> ::= `{' <Stmt>* `}' | <FunCall> `;' | `return' [\,<Exp>\,] `;'
    \alt <id> <Field> `=' <Exp> `;'
    \alt `if' `(' <Exp> `)' <Stmt> [\,`else' <Stmt>\,]
    \alt `while' `(' <Exp> `)' <Stmt>

    <Exp> ::= <int> | <char> | <bool> | `[]' | <id> <Field> | <FunCall>
    \alt <Op1> <Exp> | <Exp> <Op2> <Exp>
    \alt `(' <Exp> `)' | `(' <Exp> `,' <Exp> `)'

    <Field> ::= [\,<Field> `.' (\,`hd' | `tl' | `fst' | `snd'\,)\,]

    <FunCall> ::= <id> `(' [\,<ActArgs>\,] `)'

    <ActArgs> ::= <Exp> [\,`,' <ActArgs>\,]

    <Op1> ::= `!' | `-'

    <Op2> ::=  `*' | `/' | `%' | `+' | `-'
    \alt `<' | `>' | `<=' | `>=' | `==' | `!='
    \alt `&&' | `||' | `:'

    <int> ::= [\,`-'\,] <digit>+

    <char> ::= `\'' <any> `\''

    <bool> ::= `False' | `True'

    <id> ::= <alpha> (\,`_' | <alphaNum>\,)*
\end{grammar}
Note that we made some slight changes to the original grammar.
Most importantly, we added the \synt{char} rule that was missing in the assignment.

It is assumed that the \synt{digit}, \synt{any}, \synt{alpha}, and \synt{alphaNum} rules are clear; enumerating these would not be practical.

The operators should adhere to the usual precedence and associativity rules.
The following grammar is transformed from the previous one in order to achieve that.
Note that only list construction is right-associative.
\setlength{\grammarindent}{7.1em}
\begin{grammar}
    <SPL> ::= <Decl>+

    <Decl> ::= <VarDecl> | <FunDecl>

    <VarDecl> ::= <Type> <id> `=' <Exp> `;'

    <FunDecl> ::= <RetType> <id> `(' [\,<FArgs>\,] `)' `{' <VarDecl>* <Stmt>+ `}'

    <RetType> ::= <Type> | `Void'

    <Type> ::= <id> | <BasicType> | `(' <Type> `,' <Type> `)' | `[' <Type> `]'

    <BasicType> ::= `Int' | `Bool' | `Char'

    <FArgs> ::= [\,<FArgs> `,'\,] <Type> <id>

    <Stmt> ::= `{' <Stmt>* `}' | <FunCall> `;' | `return' [\,<Exp>\,] `;'
    \alt <id> <Field> `=' <Exp> `;'
    \alt `if' `(' <Exp> `)' <Stmt> [\,`else' <Stmt>\,]
    \alt `while' `(' <Exp> `)' <Stmt>

    <Exp> ::= <Exp1> [\,`:' <Exp>\,]

    <Exp1> ::= [\,<Exp1> (\,`&&' | `||'\,)\,] <Exp2>

    <Exp2> ::= [\,<Exp2> (\,`==' | `!='\,)\,] <Exp3>

    <Exp3> ::= [\,<Exp3> (\,`<' | `>' | `<=' | `>='\,)\,] <Exp4>

    <Exp4> ::= [\,<Exp4> (\,`+' | `-'\,)\,] <Exp5>

    <Exp5> ::= [\,<Exp5> (\,`*' | `/' | `%'\,)\,] <Exp6>

    <Exp6> ::= (\,`!' | `-'\,) <Exp6> | <NonOpExp>

    <NonOpExp> ::= <int> | <char> | <bool> | `[]' | <id> <Field> | <FunCall>
    \alt `(' <Exp> `)' | `(' <Exp> `,' <Exp> `)'

    <Field> ::= [\,<Field> `.' (\,`hd' | `tl' | `fst' | `snd'\,)\,]

    <FunCall> ::= <id> `(' [\,<ActArgs>\,] `)'

    <ActArgs> ::= <Exp> [\,`,' <ActArgs>\,]

    <int> ::= [\,`-'\,] <digit>+

    <char> ::= `\'' <any> `\''

    <bool> ::= `False' | `True'

    <id> ::= <alpha> (\,`_' | <alphaNum>\,)*
\end{grammar}

We will now eliminate left recursion by first removing empty productions.
These are only present in the \synt{Field} rule, which we easily adapt.
\begin{grammar}
    <SPL> ::= <Decl>+

    <Decl> ::= <VarDecl> | <FunDecl>

    <VarDecl> ::= <Type> <id> `=' <Exp> `;'

    <FunDecl> ::= <RetType> <id> `(' [\,<FArgs>\,] `)' `{' <VarDecl>* <Stmt>+ `}'

    <RetType> ::= <Type> | `Void'

    <Type> ::= <id> | <BasicType> | `(' <Type> `,' <Type> `)' | `[' <Type> `]'

    <BasicType> ::= `Int' | `Bool' | `Char'

    <FArgs> ::= [\,<FArgs> `,'\,] <Type> <id>

    <Stmt> ::= `{' <Stmt>* `}' | <FunCall> `;' | `return' [\,<Exp>\,] `;'
    \alt <id> [\,<Field>\,] `=' <Exp> `;'
    \alt `if' `(' <Exp> `)' <Stmt> [\,`else' <Stmt>\,]
    \alt `while' `(' <Exp> `)' <Stmt>

    <Exp> ::= <Exp1> [\,`:' <Exp>\,]

    <Exp1> ::= [\,<Exp1> (\,`&&' | `||'\,)\,] <Exp2>

    <Exp2> ::= [\,<Exp2> (\,`==' | `!='\,)\,] <Exp3>

    <Exp3> ::= [\,<Exp3> (\,`<' | `>' | `<=' | `>='\,)\,] <Exp4>

    <Exp4> ::= [\,<Exp4> (\,`+' | `-'\,)\,] <Exp5>

    <Exp5> ::= [\,<Exp5> (\,`*' | `/' | `%'\,)\,] <Exp6>

    <Exp6> ::= (\,`!' | `-'\,) <Exp6> | <NonOpExp>

    <NonOpExp> ::= <int> | <char> | <bool> | `[]' | <id> [\,<Field>\,] | <FunCall>
    \alt `(' <Exp> `)' | `(' <Exp> `,' <Exp> `)'

    <Field> ::= [\,<Field>\,] `.' (\,`hd' | `tl' | `fst' | `snd'\,)

    <FunCall> ::= <id> `(' [\,<ActArgs>\,] `)'

    <ActArgs> ::= <Exp> [\,`,' <ActArgs>\,]

    <int> ::= [\,`-'\,] <digit>+

    <char> ::= `\'' <any> `\''

    <bool> ::= `False' | `True'

    <id> ::= <alpha> (\,`_' | <alphaNum>\,)*
\end{grammar}

Fortunately there are no cycles in our current grammar, so we skip to the last step: eliminating immediate left recursion.
\begin{grammar}
    <SPL> ::= <Decl>+

    <Decl> ::= <VarDecl> | <FunDecl>

    <VarDecl> ::= <Type> <id> `=' <Exp> `;'

    <FunDecl> ::= <RetType> <id> `(' [\,<FArgs>\,] `)' `{' <VarDecl>* <Stmt>+ `}'

    <RetType> ::= <Type> | `Void'

    <Type> ::= <id> | <BasicType> | `(' <Type> `,' <Type> `)' | `[' <Type> `]'

    <BasicType> ::= `Int' | `Bool' | `Char'

    <FArgs> ::= <Type> <id> [\,`,' <FArgs>\,]

    <Stmt> ::= `{' <Stmt>* `}' | <FunCall> `;' | `return' [\,<Exp>\,] `;'
    \alt <id> [\,<Field>\,] `=' <Exp> `;'
    \alt `if' `(' <Exp> `)' <Stmt> [\,`else' <Stmt>\,]
    \alt `while' `(' <Exp> `)' <Stmt>

    <Exp> ::= <Exp1> [\,`:' <Exp>\,]

    <Exp1> ::= <Exp2> (\,(\,`&&' | `||'\,) <Exp2>\,)*

    <Exp2> ::= <Exp3> (\,(\,`==' | `!='\,) <Exp3>\,)*

    <Exp3> ::= <Exp4> (\,(\,`<' | `>' | `<=' | `>='\,) <Exp4>\,)*

    <Exp4> ::= <Exp5> (\,(\,`+' | `-'\,) <Exp5>\,)*

    <Exp5> ::= <Exp6> (\,(\,`*' | `/' | `%'\,) <Exp6>\,)*

    <Exp6> ::= (\,`!' | `-'\,) <Exp6> | <NonOpExp>

    <NonOpExp> ::= <int> | <char> | <bool> | `[]' | <id> [\,<Field>\,] | <FunCall>
    \alt `(' <Exp> `)' | `(' <Exp> `,' <Exp> `)'

    <Field> ::= `.' (\,`hd' | `tl' | `fst' | `snd'\,) [\,<Field>\,]

    <FunCall> ::= <id> `(' [\,<ActArgs>\,] `)'

    <ActArgs> ::= <Exp> [\,`,' <ActArgs>\,]

    <int> ::= [\,`-'\,] <digit>+

    <char> ::= `\'' <any> `\''

    <bool> ::= `False' | `True'

    <id> ::= <alpha> (\,`_' | <alphaNum>\,)*
\end{grammar}
There is still some ambiguity in this grammar, but we solve that in our specific implementation by favoring certain options over others.

\end{document}
