\documentclass{article}

\usepackage[latin1]{inputenc}
\usepackage[english]{babel}
\usepackage{mathtools}
\usepackage{amsfonts}
\usepackage{amssymb}
\usepackage[nounderscore]{syntax}

\setlength{\grammarparsep}{0pt}
\renewcommand{\syntleft}{\normalfont\itshape}
\renewcommand{\syntright}{}

\title{SPL Parser Grammar}
\author{Tom Sanders and Gerco van Heerdt}

\begin{document}

\maketitle

We recall first the original grammar of the Simple Programming Language\footnote{%
    tl;dr: the final grammar is at the end of this document.
}.
\setlength{\grammarindent}{6.8em}
\begin{grammar}
    <SPL> ::= <Decl>+

    <Decl> ::= <VarDecl> | <FunDecl>

    <VarDecl> ::= <Type> <id> `=' <Exp> `;'

    <FunDecl> ::= <RetType> <id> `(' [\,<FArgs>\,] `)' `{' <VarDecl>* <Stmt>+ `}'

    <RetType> ::= <Type> | `Void'

    <Type> ::= <id> | <BasicType> | `(' <Type> `,' <Type> `)' | `[' <Type> `]'

    <BasicType> ::= `Int' | `Bool' | `Char'

    <FArgs> ::= [\,<FArgs> `,'\,] <Type> <id>

    <Stmt> ::= `{' <Stmt>* `}' | <FunCall> `;' | `return' [\,<Exp>\,] `;'
    \alt <id> <Field> `=' <Exp> `;'
    \alt `if' `(' <Exp> `)' <Stmt> [\,`else' <Stmt>\,]
    \alt `while' `(' <Exp> `)' <Stmt>

    <Exp> ::= <int> | <char> | `False' | `True' | `[]' | <id> <Field>
    \alt <FunCall> | <Op1> <Exp> | <Exp> <Op2> <Exp>
    \alt `(' <Exp> `)' | `(' <Exp> `,' <Exp> `)'

    <Field> ::= [\,<Field> `.' (\,`hd' | `tl' | `fst' | `snd'\,)\,]

    <FunCall> ::= <id> `(' [\,<ActArgs>\,] `)'

    <ActArgs> ::= <Exp> [\,`,' <ActArgs>\,]

    <Op1> ::= `!' | `-'

    <Op2> ::=  `*' | `/' | `%' | `+' | `-'
    \alt `<' | `>' | `<=' | `>=' | `==' | `!='
    \alt `&&' | `||' | `:'

    <int> ::= [\,`-'\,] <digit>+

    <id> ::= <alpha> (\,`_' | <alphaNum>\,)*
\end{grammar}

The operators should adhere to the usual precedence and associativity rules.
The following grammar is transformed from the previous one in order to achieve that.
Note that only list construction is right-associative.
\setlength{\grammarindent}{7.1em}
\begin{grammar}
    <SPL> ::= <Decl>+

    <Decl> ::= <VarDecl> | <FunDecl>

    <VarDecl> ::= <Type> <id> `=' <Exp> `;'

    <FunDecl> ::= <RetType> <id> `(' [\,<FArgs>\,] `)' `{' <VarDecl>* <Stmt>+ `}'

    <RetType> ::= <Type> | `Void'

    <Type> ::= <id> | <BasicType> | `(' <Type> `,' <Type> `)' | `[' <Type> `]'

    <BasicType> ::= `Int' | `Bool' | `Char'

    <FArgs> ::= [\,<FArgs> `,'\,] <Type> <id>

    <Stmt> ::= `{' <Stmt>* `}' | <FunCall> `;' | `return' [\,<Exp>\,] `;'
    \alt <id> <Field> `=' <Exp> `;'
    \alt `if' `(' <Exp> `)' <Stmt> [\,`else' <Stmt>\,]
    \alt `while' `(' <Exp> `)' <Stmt>

    <Exp> ::= <OpExp> | <NonOpExp>

    <OpExp> ::= <OpExp1> [\,`:' <OpExp>\,]

    <OpExp1> ::= [\,<OpExp1> (\,`&&' | `||'\,)\,] <OpExp2>

    <OpExp2> ::= [\,<OpExp2> (\,`==' | `!='\,)\,] <OpExp3>

    <OpExp3> ::= [\,<OpExp3> (\,`<' | `>' | `<=' | `>='\,)\,] <OpExp4>

    <OpExp4> ::= [\,<OpExp4> (\,`+' | `-'\,)\,] <OpExp5>

    <OpExp5> ::= [\,<OpExp5> (\,`*' | `/' | `%'\,)\,] <OpExp6>

    <OpExp6> ::= (\,`!' | `-'\,) <OpExp6> | <NonOpExp>

    <NonOpExp> ::= <int> | <char> | `False' | `True' | `[]' | <id> <Field>
    \alt <FunCall> | `(' <Exp> `)' | `(' <Exp> `,' <Exp> `)'

    <Field> ::= [\,<Field> `.' (\,`hd' | `tl' | `fst' | `snd'\,)\,]

    <FunCall> ::= <id> `(' [\,<ActArgs>\,] `)'

    <ActArgs> ::= <Exp> [\,`,' <ActArgs>\,]

    <int> ::= [\,`-'\,] <digit>+

    <id> ::= <alpha> (\,`_' | <alphaNum>\,)*
\end{grammar}

We will now eliminate left recursion.
The first step is to remove empty productions.
These are only present in the \synt{Field} rule, which we easily adapt.
\begin{grammar}
    <SPL> ::= <Decl>+

    <Decl> ::= <VarDecl> | <FunDecl>

    <VarDecl> ::= <Type> <id> `=' <Exp> `;'

    <FunDecl> ::= <RetType> <id> `(' [\,<FArgs>\,] `)' `{' <VarDecl>* <Stmt>+ `}'

    <RetType> ::= <Type> | `Void'

    <Type> ::= <id> | <BasicType> | `(' <Type> `,' <Type> `)' | `[' <Type> `]'

    <BasicType> ::= `Int' | `Bool' | `Char'

    <FArgs> ::= [\,<FArgs> `,'\,] <Type> <id>

    <Stmt> ::= `{' <Stmt>* `}' | <FunCall> `;' | `return' [\,<Exp>\,] `;'
    \alt <id> [\,<Field>\,] `=' <Exp> `;'
    \alt `if' `(' <Exp> `)' <Stmt> [\,`else' <Stmt>\,]
    \alt `while' `(' <Exp> `)' <Stmt>

    <Exp> ::= <OpExp> | <NonOpExp>

    <OpExp> ::= <OpExp1> [\,`:' <OpExp>\,]

    <OpExp1> ::= [\,<OpExp1> (\,`&&' | `||'\,)\,] <OpExp2>

    <OpExp2> ::= [\,<OpExp2> (\,`==' | `!='\,)\,] <OpExp3>

    <OpExp3> ::= [\,<OpExp3> (\,`<' | `>' | `<=' | `>='\,)\,] <OpExp4>

    <OpExp4> ::= [\,<OpExp4> (\,`+' | `-'\,)\,] <OpExp5>

    <OpExp5> ::= [\,<OpExp5> (\,`*' | `/' | `%'\,)\,] <OpExp6>

    <OpExp6> ::= (\,`!' | `-'\,) <OpExp6> | <NonOpExp>

    <NonOpExp> ::= <int> | <char> | `False' | `True' | `[]' | <id> [\,<Field>\,]
    \alt <FunCall> | `(' <Exp> `)' | `(' <Exp> `,' <Exp> `)'

    <Field> ::= [\,<Field>\,] `.' (\,`hd' | `tl' | `fst' | `snd'\,)

    <FunCall> ::= <id> `(' [\,<ActArgs>\,] `)'

    <ActArgs> ::= <Exp> [\,`,' <ActArgs>\,]

    <int> ::= [\,`-'\,] <digit>+

    <id> ::= <alpha> (\,`_' | <alphaNum>\,)*
\end{grammar}

\end{document}
