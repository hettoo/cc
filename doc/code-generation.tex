\documentclass[a4paper]{article}

\usepackage[latin1]{inputenc}
\usepackage[english]{babel}
\usepackage{mathtools}
\usepackage{amsfonts}
\usepackage{amssymb}
\usepackage[nounderscore]{syntax}
\usepackage{bussproofs}

\setlength{\grammarparsep}{0pt}
\renewcommand{\syntleft}{\normalfont\itshape}
\renewcommand{\syntright}{}

\title{SPL Code Generation}
\author{Tom Sanders and Gerco van Heerdt}

\begin{document}

\maketitle

\section{Syntax}

Recall our final version of the SPL grammar.
\setlength{\grammarindent}{7.1em}
\begin{grammar}
    <SPL> ::= <Decl>+

    <Decl> ::= <VarDecl> | <FunDecl>

    <VarDecl> ::= <Type> <id> `=' <Exp> `;'

    <FunDecl> ::= <RetType> <id> `(' [\,<FArgs>\,] `)' `{' <VarDecl>* <Stmt>+ `}'

    <RetType> ::= <Type> | `Void'

    <Type> ::= <id> | <BasicType> | `(' <Type> `,' <Type> `)' | `[' <Type> `]'

    <BasicType> ::= `Int' | `Bool' | `Char'

    <FArgs> ::= <Type> <id> [\,`,' <FArgs>\,]

    <Stmt> ::= `{' <Stmt>* `}' | <StmtId> `;' | `return' [\,<Exp>\,] `;'
    \alt `if' `(' <Exp> `)' <Stmt> [\,`else' <Stmt>\,]
    \alt `while' `(' <Exp> `)' <Stmt>

    <StmtId> ::= <id> (\,[\,<Field>\,] `=' <Exp> | `(' [\,<ActArgs>\,] `)'\,)

    <Exp> ::= <Exp1> [\,`:' <Exp>\,]

    <Exp1> ::= <Exp2> (\,(\,`&&' | `||'\,) <Exp2>\,)*

    <Exp2> ::= <Exp3> (\,(\,`==' | `!='\,) <Exp3>\,)*

    <Exp3> ::= <Exp4> (\,(\,`<' | `>' | `<=' | `>='\,) <Exp4>\,)*

    <Exp4> ::= <Exp5> (\,(\,`+' | `-'\,) <Exp5>\,)*

    <Exp5> ::= <Exp6> (\,(\,`*' | `/' | `%'\,) <Exp6>\,)*

    <Exp6> ::= (\,`!' | `-'\,) <Exp6> | <NonOpExp>

    <NonOpExp> ::= <int> | <char> | <bool> | `[]' | <ExpId>
    \alt `(' <Exp> `)' | `(' <Exp> `,' <Exp> `)'

    <ExpId> ::= <id> (\,[\,<Field>\,] | `(' [\,<ActArgs>\,] `)'\,)

    <Field> ::= `.' (\,`hd' | `tl' | `fst' | `snd'\,) [\,<Field>\,]

    <ActArgs> ::= <Exp> [\,`,' <ActArgs>\,]

    <int> ::= [\,`-'\,] <digit>+

    <char> ::= `\'' <any> `\''

    <bool> ::= `False' | `True'

    <id> ::= <alpha> (\,`_' | <alphaNum>\,)*
\end{grammar}

What matters for typing is not the syntax used to recognize a program, but rather the syntax used to store it (the relation with the above grammar should be clear).
For instance, the grammar describing our expressions is shown below.
\setlength{\grammarindent}{4.1em}
\begin{grammar}
    <Exp> ::= Int $i$ | Bool $b$ | Char $c$ | Nil | $\ominus$<Exp> | <Exp> $\otimes$ <Exp>
    \alt (\,<Exp>, <Exp>\,) | Var <id> <Field>* | Fun <id> <Exp>*
\end{grammar}
Here $\ominus$ stands for a unary operator, and $\otimes$ can be any binary operator.

\section{Typing rules}

For convenience, we use the notation that a context $\Gamma$ is made up of a variable context $\Gamma_v$ and a function context $\Gamma_f$.
Formally, $\Gamma = (\Gamma_v, \Gamma_f)$.

\begin{center}
    \AxiomC{}
    \UnaryInfC{$\Gamma \vdash \text{Int } i : \text{Int}$}
    \DisplayProof
    \qquad
    \AxiomC{}
    \UnaryInfC{$\Gamma \vdash \text{Bool } i : \text{Bool}$}
    \DisplayProof
    \qquad
    \AxiomC{}
    \UnaryInfC{$\Gamma \vdash \text{Char } i : \text{Char}$}
    \DisplayProof
\end{center}

\begin{center}
    \AxiomC{}
    \UnaryInfC{$\Gamma \vdash \text{Nil} : \forall a.\,[a]$}
    \DisplayProof
    \qquad
    \AxiomC{$\Gamma \vdash e_1 : \sigma_1$}
    \AxiomC{$\Gamma \vdash e_2 : \sigma_2$}
    \BinaryInfC{$\Gamma \vdash (e_1, e_2) : (\sigma_1, \sigma_2)$}
    \DisplayProof
    \qquad
    \AxiomC{$(i, \sigma) \in \Gamma_v$}
    \UnaryInfC{$\Gamma \vdash \text{Var } i\;[\,] : \sigma$}
    \DisplayProof
\end{center}

\begin{center}
    \AxiomC{$(i, \sigma) \in \Gamma_v$}
    \AxiomC{$\Gamma \vdash f : \sigma \to \tau$}
    \AxiomC{$((\Gamma_{v \setminus i}, i : \tau), \Gamma_f) \vdash \text{Var } i\;r : \chi$}
    \TrinaryInfC{$\Gamma \vdash \text{Var } i\;(f :: r) : \chi$}
    \DisplayProof
\end{center}
Here $\Gamma_{v \setminus i} = \{(v, t) \in \Gamma_v \mid v \neq i\}$.

Note that the above rule requires us to type field specifiers as well.
We will do this later.

\begin{center}
    \AxiomC{$\forall_{1 \le j \le n}\,\Gamma \vdash e_j : \sigma_j$}
    \AxiomC{$(i, (\sigma_1, \sigma_2, \ldots, \sigma_n) \to \tau) \in \Gamma_f$}
    \BinaryInfC{$\Gamma \vdash \text{Fun } i\;(e_1, e_2, \ldots, e_n) : \tau$}
    \DisplayProof
\end{center}

Finally, we have to specify all operators.

\begin{center}
    \AxiomC{$\Gamma \vdash e : \text{Int}$}
    \UnaryInfC{$\Gamma \vdash -e : \text{Int}$}
    \DisplayProof
    \qquad
    \AxiomC{$\Gamma \vdash e : \text{Bool}$}
    \UnaryInfC{$\Gamma \vdash\;!e : \text{Bool}$}
    \DisplayProof
    \qquad
    \AxiomC{$\Gamma \vdash e_1 : \sigma$}
    \AxiomC{$\Gamma \vdash e_2 : \sigma$}
    \BinaryInfC{$\Gamma \vdash e_1 \bullet e_2 : \sigma$}
    \DisplayProof
\end{center}
where $\bullet \in \{+, -, *, /, \%\}$.

\begin{center}
    \AxiomC{$\Gamma \vdash e : \sigma$}
    \AxiomC{$\Gamma \vdash l : [\sigma]$}
    \BinaryInfC{$\Gamma \vdash (e : l) : [\sigma]$}
    \DisplayProof
    \qquad
    \AxiomC{$\Gamma \vdash e_1 : \sigma$}
    \AxiomC{$\Gamma \vdash e_2 : \sigma$}
    \BinaryInfC{$\Gamma \vdash e_1 \diamond e_2 : \text{Bool}$}
    \DisplayProof
\end{center}
with $\diamond \in \{==, <, >, <=, >=, !\!=\}$.

\begin{center}
    \AxiomC{$\Gamma \vdash e_1 : \text{Bool}$}
    \AxiomC{$\Gamma \vdash e_2 : \text{Bool}$}
    \BinaryInfC{$\Gamma \vdash e_1 \text{ \&\& } e_2 : \text{Bool}$}
    \DisplayProof
    \qquad
    \AxiomC{$\Gamma \vdash e_1 : \text{Bool}$}
    \AxiomC{$\Gamma \vdash e_2 : \text{Bool}$}
    \BinaryInfC{$\Gamma \vdash e_1 \text{ \textbar\textbar{} } e_2 : \text{Bool}$}
    \DisplayProof
\end{center}

Field specifiers are given by the following grammar:
\setlength{\grammarindent}{4.3em}
\begin{grammar}
    <Field> ::= Head | Tail | First | Second
\end{grammar}

The types of these are given by the axioms below.
\begin{center}
    \AxiomC{}
    \UnaryInfC{$\Gamma \vdash \text{Head} : [\sigma] \to \sigma$}
    \DisplayProof
    \qquad
    \AxiomC{}
    \UnaryInfC{$\Gamma \vdash \text{Tail} : [\sigma] \to [\sigma]$}
    \DisplayProof
\end{center}
\begin{center}
    \AxiomC{}
    \UnaryInfC{$\Gamma \vdash \text{First} : (\sigma_1, \sigma_2) \to \sigma_1$}
    \DisplayProof
    \qquad
    \AxiomC{}
    \UnaryInfC{$\Gamma \vdash \text{Second} : (\sigma_1, \sigma_2) \to \sigma_2$}
    \DisplayProof
\end{center}

Finally, we have the following built-in functions, typed through context initialization:
\[
    \begin{array}{lcl}
        \mathit{isEmpty} & : & [\sigma] \to \text{Bool} \\
        \mathit{print}   & : & \sigma \to \text{Void} \\
        \mathit{read}    & : & \text{Int}
    \end{array}
\]
Here Void simply indicates the absence of a return value.
Note that we do not allow overloading or polymorphism purely on the return type, and so we choose to fix the type of the $\mathit{read}$ function to Int.
The reason for this is to avoid ambiguity: instead of as an expression a function call may appear as a statement, in which case the desired return type is not defined.
We further reject overloading when two versions of a function potentially apply to a certain same combination of argument types.

\section{Semantic Analysis}

Binding time analysis consists in determining where variables that are used in the program were bound.
We use the usual precedence rules to do this: variables defined later override earlier ones, unless they were defined in the same scope (block of code or function argument lists), which is an error.
Note that both function arguments and function bodies introduce a new scope.

Typechecking a function consists in ensuring that the returned values for all possible executions of the function (assuming just that it terminates) are compatible with the type given to it.
An execution not hitting a return statement is equivalent to it returning a variable of type Void in the end.
Of course there are no such variables, but this is in turn modeled by the return statement without parameter.

Functions may be overloaded, but it is not allowed that multiple versions of a function apply for any possible way the function can be called.
Note that the current implementation does not allow overloading the return type of a function at all.

\subsection{Test cases}

The following test cases primarily test for program instances that should be rejected by the compiler due to typing problems.
For positive tests we have mostly relied on using the example programs for the first assignment, except for specific type constructs which were not covered by these examples.
These test cases also introduce some of our design choices.
General tests beyond the scope of semantic analysis are contained within the \verb|tests/general| directory of the project and have already been described in the previous assignment.

\paragraph{tests/typing/error1.spl}
In this example we show that polymorphic returns are checked.
Calling \verb|g| with an integer yield an integer, which is too specific for the return type of \verb|f|.
    \begin{verbatim}
incompatible return type `Int' provided; expected `a' in
function f\end{verbatim}

\paragraph{tests/typing/error2.spl}
This test builds on the previous one, modifying the return type for \verb|g|.
The compiler correctly rejects this test because the return of \verb|g| is now ill-typed.
    \begin{verbatim}
incompatible return type `c' provided; expected `b' in function g\end{verbatim}

\paragraph{tests/typing/error3.spl}
Function application is performed with incorrect arity.
    \begin{verbatim}
no candidate for application of f found\end{verbatim}

\paragraph{tests/typing/error4.spl}
Function application is performed with incorrect arity.
    \begin{verbatim}
no candidate for application of f found\end{verbatim}

\paragraph{tests/typing/error5.spl}
Return type is too general
    \begin{verbatim}
assignment mismatch: expected type `Int' does not cover
given type `a' (variable y)\end{verbatim}

\paragraph{tests/typing/error6.spl}
This program describes an instance where a variable cannot be typed because of the unknown return type of \verb|read()|.
    \begin{verbatim}
free polymorphic variable x\end{verbatim}

\paragraph{tests/typing/error7.spl}
The type checker ensures a valid return for all possible code paths when control structures are used within a function body.
    \begin{verbatim}
function g may not return a value\end{verbatim}

\paragraph{tests/typing/error8.spl}
A return in a while control block is not sufficient to ensure all possible code paths have a valid return, because the loop condition is not evaluated at compile-time.
    \begin{verbatim}
function f may not return a value\end{verbatim}

\paragraph{tests/typing/error9.spl}
Variables may only be defined once in their scope.
    \begin{verbatim}
redefined variable x\end{verbatim}

\paragraph{tests/typing/error10.spl}
Functions may only be defined once.
    \begin{verbatim}
redefined function f\end{verbatim}

\paragraph{tests/typing/error11.spl}
Function arguments may only be defined once.
    \begin{verbatim}
duplicate formal argument n for function f\end{verbatim}

\paragraph{tests/typing/error12.spl}
Variables may not be used outside of their scope.
    \begin{verbatim}
undeclared entity a\end{verbatim}

\paragraph{tests/typing/error13.spl}
Functions must be called with the right arity.
    \begin{verbatim}
no candidate for application of f found\end{verbatim}

\paragraph{tests/typing/error14.spl}
Function overloading instances must have mutually exclusive types.
    \begin{verbatim}
redefined function id\end{verbatim}

\paragraph{tests/typing/error15.spl}
The main function may not be polymorphic.
    \begin{verbatim}
invalid main function\end{verbatim}

\paragraph{tests/typing/error16.spl}
The main function may not accept arguments.
    \begin{verbatim}
invalid main function\end{verbatim}

\paragraph{tests/typing/error17.spl}
Overloading must not be purely on the return type.
    \begin{verbatim}
redefined function f\end{verbatim}

\paragraph{tests/typing/error18.spl}
The arguments for the application of a polymorphic function must match its scheme.
    \begin{verbatim}
application mismatch: expected type `(?1, ?1)' does not cover
given type `(Int, Char)'\end{verbatim}

\paragraph{tests/typing/overloading1.spl}\mbox{}\\ % TODO: :(
Overloaded function applications are allowed.
    \begin{verbatim}
Void f () {
    f((0 @ Int));
}

Void f (Int a) {
}\end{verbatim}

\paragraph{tests/typing/overloading2.spl}\mbox{}\\ % TODO: :(
Overloaded function applications are allowed.
    \begin{verbatim}
Void f (Char a) {
    f((0 @ Int));
}

Void f (Int a) {
}\end{verbatim}

\paragraph{tests/typing/scope1.spl}
Local variables hide local variables.
    \begin{verbatim}
Int a = (0 @ Int);

Char f (Char a) {
    return (a @ Char);
}\end{verbatim}

\paragraph{tests/typing/scope2.spl}
Local variables hide function arguments.
    \begin{verbatim}
Int f (Char a) {
    Int a = (0 @ Int);

    return (a @ Int);
}\end{verbatim}

\section{Compilation Schemes}

We use \emph{by-value} assignments.
For tuples and lists this value is a pointer to objects on the heap that gets passed around, but in our implementation the heap is immutable.
Thus, an object passed to a function will never actually be modified by that function.

All objects are represented on the stack with a single value.
As mentioned above, tuples and lists are stored on the heap.
In the case of a tuple the first object is stored first, followed by the second object; in the case of a list we store the tail first and the head second.
The empty list is represented by a null pointer.

\subsection{Test Cases}

% TODO

\section{Build Instructions}

Building the project requires a modern Haskell compiler and a Make-like system.
Running \verb|make| in the project root will build the compiler in the \verb|build| folder.
Tests can either be run manually by feeding SPL source files into the compiler with \verb|build/compiler < program.spl| or automatically by calling \verb|make tests| in the project root to run a set of predefined tests.
If no problems are encountered, the compiler generates the file \verb|a.ssm| containing the compiled program.

\end{document}
