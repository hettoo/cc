\documentclass[a4paper]{article}

\usepackage[latin1]{inputenc}
\usepackage[english]{babel}
\usepackage{mathtools}
\usepackage{amsfonts}
\usepackage{amssymb}
\usepackage[nounderscore]{syntax}
\usepackage{bussproofs}

\setlength{\grammarparsep}{0pt}
\renewcommand{\syntleft}{\normalfont\itshape}
\renewcommand{\syntright}{}

\title{SPL Semantic Analysis}
\author{Tom Sanders and Gerco van Heerdt}

\begin{document}

\maketitle

Recall our final version of the SPL grammar.
\setlength{\grammarindent}{7.1em}
\begin{grammar}
    <SPL> ::= <Decl>+

    <Decl> ::= <VarDecl> | <FunDecl>

    <VarDecl> ::= <Type> <id> `=' <Exp> `;'

    <FunDecl> ::= <RetType> <id> `(' [\,<FArgs>\,] `)' `{' <VarDecl>* <Stmt>+ `}'

    <RetType> ::= <Type> | `Void'

    <Type> ::= <id> | <BasicType> | `(' <Type> `,' <Type> `)' | `[' <Type> `]'

    <BasicType> ::= `Int' | `Bool' | `Char'

    <FArgs> ::= <Type> <id> [\,`,' <FArgs>\,]

    <Stmt> ::= `{' <Stmt>* `}' | <StmtId> `;' | `return' [\,<Exp>\,] `;'
    \alt `if' `(' <Exp> `)' <Stmt> [\,`else' <Stmt>\,]
    \alt `while' `(' <Exp> `)' <Stmt>

    <StmtId> ::= <id> (\,[\,<Field>\,] `=' <Exp> | `(' [\,<ActArgs>\,] `)'\,)

    <Exp> ::= <Exp1> [\,`:' <Exp>\,]

    <Exp1> ::= <Exp2> (\,(\,`&&' | `||'\,) <Exp2>\,)*

    <Exp2> ::= <Exp3> (\,(\,`==' | `!='\,) <Exp3>\,)*

    <Exp3> ::= <Exp4> (\,(\,`<' | `>' | `<=' | `>='\,) <Exp4>\,)*

    <Exp4> ::= <Exp5> (\,(\,`+' | `-'\,) <Exp5>\,)*

    <Exp5> ::= <Exp6> (\,(\,`*' | `/' | `%'\,) <Exp6>\,)*

    <Exp6> ::= (\,`!' | `-'\,) <Exp6> | <NonOpExp>

    <NonOpExp> ::= <int> | <char> | <bool> | `[]' | <ExpId>
    \alt `(' <Exp> `)' | `(' <Exp> `,' <Exp> `)'

    <ExpId> ::= <id> (\,[\,<Field>\,] | `(' [\,<ActArgs>\,] `)'\,)

    <Field> ::= `.' (\,`hd' | `tl' | `fst' | `snd'\,) [\,<Field>\,]

    <ActArgs> ::= <Exp> [\,`,' <ActArgs>\,]

    <int> ::= [\,`-'\,] <digit>+

    <char> ::= `\'' <any> `\''

    <bool> ::= `False' | `True'

    <id> ::= <alpha> (\,`_' | <alphaNum>\,)*
\end{grammar}

What matters, however, is not the syntax used to recognize a program, but rather the syntax used to store it (the relation with the above grammar should be clear).
For instance, the grammar describing our expressions is shown below.
\setlength{\grammarindent}{4.1em}
\begin{grammar}
    <Exp> ::= Int $i$ | Bool $b$ | Char $c$ | Nil | $\ominus$<Exp> | <Exp> $\otimes$ <Exp>
    \alt (\,<Exp>, <Exp>\,) | Var <id> <Field>* | Fun <id> <Exp>*
\end{grammar}
Here $\ominus$ stands for a unary operator, and $\otimes$ can be any binary operator.

For convenience, we use the notation that a context $\Gamma$ is made up of a variable context $\Gamma_v$ and a function context $\Gamma_f$. Formally, $\Gamma = (\Gamma_v, \Gamma_f)$.

\begin{center}
    \AxiomC{}
    \UnaryInfC{$\Gamma \vdash \text{Int } i : \text{Int}$}
    \DisplayProof
    \qquad
    \AxiomC{}
    \UnaryInfC{$\Gamma \vdash \text{Bool } i : \text{Bool}$}
    \DisplayProof
    \qquad
    \AxiomC{}
    \UnaryInfC{$\Gamma \vdash \text{Char } i : \text{Char}$}
    \DisplayProof
\end{center}

\begin{center}
    \AxiomC{}
    \UnaryInfC{$\Gamma \vdash \text{Nil} : \forall a.\,[a]$}
    \DisplayProof
    \qquad
    \AxiomC{$\Gamma \vdash e_1 : \sigma_1$}
    \AxiomC{$\Gamma \vdash e_2 : \sigma_2$}
    \BinaryInfC{$\Gamma \vdash (e_1, e_2) : (\sigma_1, \sigma_2)$}
    \DisplayProof
    \qquad
    \AxiomC{$(i, \sigma) \in \Gamma_v$}
    \UnaryInfC{$\Gamma \vdash \text{Var } i\;[\,] : \sigma$}
    \DisplayProof
\end{center}

\begin{center}
    \AxiomC{$(i, \sigma) \in \Gamma_v$}
    \AxiomC{$\Gamma \vdash f : \sigma \to \tau$}
    \AxiomC{$((\Gamma_{v \setminus i}, i : \tau), \Gamma_f) \vdash \text{Var } i\;r : \chi$}
    \TrinaryInfC{$\Gamma \vdash \text{Var } i\;(f :: r) : \chi$}
    \DisplayProof
\end{center}
Here $\Gamma_{v \setminus i} = \{(v, t) \in \Gamma_v \mid v \neq i\}$.

Note that the above rule requires us to type field specifiers as well.
We will do this later.

\begin{center}
    \AxiomC{$\forall_{1 \le j \le n}\,\Gamma \vdash e_j : \sigma_j$}
    \AxiomC{$(i, (\sigma_1, \sigma_2, \ldots, \sigma_n) \to \tau) \in \Gamma_f$}
    \BinaryInfC{$\Gamma \vdash \text{Fun } i\;(e_1, e_2, \ldots, e_n) : \tau$}
    \DisplayProof
\end{center}

Finally, we have to specify all operators.

\begin{center}
    \AxiomC{$\Gamma \vdash e : \text{Int}$}
    \UnaryInfC{$\Gamma \vdash -e : \text{Int}$}
    \DisplayProof
    \qquad
    \AxiomC{$\Gamma \vdash e : \text{Bool}$}
    \UnaryInfC{$\Gamma \vdash\;!e : \text{Bool}$}
    \DisplayProof
    \qquad
    \AxiomC{$\Gamma \vdash e_1 : \sigma$}
    \AxiomC{$\Gamma \vdash e_2 : \sigma$}
    \BinaryInfC{$\Gamma \vdash e_1 \bullet e_2 : \sigma$}
    \DisplayProof
\end{center}
where $\bullet \in \{+, -, *, /, \%\}$.

\begin{center}
    \AxiomC{$\Gamma \vdash e : \sigma$}
    \AxiomC{$\Gamma \vdash l : [\sigma]$}
    \BinaryInfC{$\Gamma \vdash (e : l) : [\sigma]$}
    \DisplayProof
    \qquad
    \AxiomC{$\Gamma \vdash e_1 : \sigma$}
    \AxiomC{$\Gamma \vdash e_2 : \sigma$}
    \BinaryInfC{$\Gamma \vdash e_1 \diamond e_2 : \text{Bool}$}
    \DisplayProof
\end{center}
with $\diamond \in \{==, <, >, <=, >=, !\!=\}$.

\begin{center}
    \AxiomC{$\Gamma \vdash e_1 : \text{Bool}$}
    \AxiomC{$\Gamma \vdash e_2 : \text{Bool}$}
    \BinaryInfC{$\Gamma \vdash e_1 \text{ \&\& } e_2 : \text{Bool}$}
    \DisplayProof
    \qquad
    \AxiomC{$\Gamma \vdash e_1 : \text{Bool}$}
    \AxiomC{$\Gamma \vdash e_2 : \text{Bool}$}
    \BinaryInfC{$\Gamma \vdash e_1 \text{ \textbar\textbar{} } e_2 : \text{Bool}$}
    \DisplayProof
\end{center}

The built-in functions will be typed through the initialization of the context.

Field specifiers are given by the following grammar:
\setlength{\grammarindent}{4.3em}
\begin{grammar}
    <Field> ::= Head | Tail | First | Second
\end{grammar}

The types of these are given by the axioms below.
\begin{center}
    \AxiomC{}
    \UnaryInfC{$\Gamma \vdash \text{Head} : [\sigma] \to \sigma$}
    \DisplayProof
    \qquad
    \AxiomC{}
    \UnaryInfC{$\Gamma \vdash \text{Tail} : [\sigma] \to [\sigma]$}
    \DisplayProof
\end{center}
\begin{center}
    \AxiomC{}
    \UnaryInfC{$\Gamma \vdash \text{First} : (\sigma_1, \sigma_2) \to \sigma_1$}
    \DisplayProof
    \qquad
    \AxiomC{}
    \UnaryInfC{$\Gamma \vdash \text{Second} : (\sigma_1, \sigma_2) \to \sigma_2$}
    \DisplayProof
\end{center}

Now consider our syntax of statements.
\setlength{\grammarindent}{4.1em}
\begin{grammar}
    <Stmt> ::= Stmts <Stmt>* | FunCall <id> <Exp>* | Return [\,Exp\,] | Assign <id> <Field>*
    \alt VarDecl <Type> <id> <Exp> | FunDecl <Type> <id> (\,<Type>, <id>\,)* <Stmt>
    \alt If <Exp> <Stmt> [\,<Stmt>\,] | While <Exp> <Stmt>
\end{grammar}

Binding time analysis consists in determining where variables that are used in the program were bound.
We use the usual precedence rules to do this: variables defined later override earlier ones, unless they were defined in the same scope (block of code or function parameter lists), which is an error.
Note the more general formulation here resulting from our more general stored statement syntax.

Typechecking a function consists in ensuring that the returned values for all possible executions of the function (assuming just that it terminates) are compatible with the type given to it.
An execution not hitting a return statement is equivalent to it returning a variable of type Void in the end.
Of course there are no such variables, but this is in turn modeled by the empty return statement.

We decorate the abstract syntax tree with type information for expressions and return type information for reachable statements in functions.
For now this full tree is printed in order to trigger the errors detected by the decoration procedures.

\end{document}
