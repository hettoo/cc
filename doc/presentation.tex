\documentclass[12pt]{beamer}

\title{SPL with Custom Data Types}
\author{Gerco van Heerdt}

\usepackage[english]{babel}
\usepackage{lmodern}
\usepackage{fixltx2e}
\usepackage{array}

\setbeamertemplate{frametitle continuation}{}
\setbeamertemplate{navigation symbols}{}
\setbeamertemplate{footline}[frame number]

\newcommand{\mcut}[1]{\ifmin\\\quad #1\else#1\fi}
\newcommand{\mfset}[2]{
    \gdef\multiframetitle{#1}
    \gdef\multiframesubtitle{#2}
}
\newenvironment{mframe}{
    \begin{frame}
        \frametitle{\multiframetitle}
        \framesubtitle{\multiframesubtitle}
}{
    \end{frame}
}

\setlength{\parskip}{\medskipamount}

\begin{document}

\frame{\titlepage}

\begin{frame}[fragile]
    \frametitle{Example}
    \framesubtitle{Bool}
    \begin{verbatim}
data Bool = True | False;\end{verbatim}
    In declarations, custom types are put between slashes:
    \begin{verbatim}
\Bool/ b = True;\end{verbatim}
\end{frame}

\begin{frame}[fragile]
    \frametitle{Example}
    \frametitle{Lists and Tuples}
    \begin{verbatim}
data List a = Nil | Cons(a hd, \List a/ tl);
data Tuple a b = Tuple(a fst, b snd);\end{verbatim}
    \verb+[a]+ is now just a synonym for \verb+\List a/+ and \verb+(a, b)+ for \verb+\Tuple a b/+
\end{frame}

\begin{frame}[fragile]
    \frametitle{Constructors}
    \begin{verbatim}
data List a = Nil | Cons(a hd, \List a/ tl);\end{verbatim}
    The assignment
    \begin{verbatim}
[Int] l = 2 : [];\end{verbatim}
    is equivalent to
    \begin{verbatim}
\List Int/ l = Cons(2, Nil);\end{verbatim}
\end{frame}

\begin{frame}[fragile]
    \frametitle{Generalized Field Accessors}
    \begin{verbatim}
data List a = Nil | Cons(a hd, \List a/ tl);\end{verbatim}
    Field names automatically become accessors:
    \begin{verbatim}
\List Int/ l = Cons(2, Nil);
print(l.hd);\end{verbatim}
    prints 2

    Setting a field has also been generalized.
\end{frame}

\begin{frame}[fragile]
    \frametitle{Duplicate Field Names}
    \begin{verbatim}
data T1 = A(Int x, Int y);
data T2 = B(Int y, Int x);

Void main() {
    \T1/ a = A(2, 3);
    \T2/ b = B(5, 7);
    print(a.x);
    print(a.y);
    print(b.y);
    print(b.x);
    a.y = 11;
    b.x = 13;
    print(a.y);
    print(b.x);
}\end{verbatim}
\end{frame}

\begin{frame}[fragile]
    \frametitle{Errors}
    \framesubtitle{Invalid Duplicate Field Names}
    \begin{verbatim}
data T = A(Int field) | B(Bool field);\end{verbatim}
\end{frame}

\begin{frame}[fragile]
    \frametitle{Errors}
    \framesubtitle{Duplicate Constructor Names}
    \begin{verbatim}
data T1 = A;
data T2 = A;\end{verbatim}
    \begin{verbatim}
data T = B | B;\end{verbatim}
\end{frame}

\begin{frame}[fragile]
    \frametitle{Errors}
    \framesubtitle{Duplicate Type Names}
    \begin{verbatim}
data T = A;
data T = B;\end{verbatim}
\end{frame}

\begin{frame}
    \frametitle{Type Checking}
    \begin{itemize}
        \item Constructor application is checked like function application
        \item Field accessors are like functions from the custom type to the field type
    \end{itemize}
\end{frame}

\begin{frame}[fragile]
    \frametitle{Constructor Matching}
    \begin{verbatim}
data List a = Nil | Cons(a hd, \List a/ tl);\end{verbatim}
    \begin{verbatim}
\List Int/ l = Cons(2, Nil);
case l {
    Nil { print(0); }
    Cons { print(l.hd); }
}\end{verbatim}
    No actual pattern matching, but field accessors alleviate this.
\end{frame}

\begin{frame}[fragile]
    \frametitle{Errors}
    \framesubtitle{Invalid Constructor}
    \begin{verbatim}
data T1 = A;
data T2 = B;

Void main() {
    case A() {
        B {}
    }
}\end{verbatim}
\end{frame}

\begin{frame}[fragile]
    \frametitle{Errors}
    \framesubtitle{Duplicate Constructor}
    \begin{verbatim}
data T = A;

Void main() {
    case A() {
        A {}
        A {}
    }
}\end{verbatim}
\end{frame}

\begin{frame}[fragile]
    \frametitle{Errors}
    \framesubtitle{Non-Custom Type}
    \begin{verbatim}
Void main() {
    case 5 {}
}\end{verbatim}
\end{frame}

\begin{frame}[fragile]
    \frametitle{Errors}
    \framesubtitle{Return Value Not Guaranteed}
    \begin{verbatim}
data T = A | B;

Bool test(\T/ a) {
    case a {
        A { return True; }
    }
}\end{verbatim}
\end{frame}

\begin{frame}[fragile]
    \frametitle{Errors}
    \framesubtitle{Return Value Not Guaranteed}
    \begin{verbatim}
data T = A | B;

Bool test(\T/ a) {
    case a {
        A { }
        B { return True; }
    }
}\end{verbatim}
\end{frame}

\begin{frame}[fragile]
    \frametitle{Example}
    \framesubtitle{Printing a List as a Tree}
    \begin{verbatim}
data Tree a = Leaf |
    Node(a val, \Tree a/ left, \Tree a/ right);

\Tree a/ fromList([a] l) {
    ([a], [a]) t;
    case l {
        Nil { return Leaf; }
        Cons {
            t = split(l.tl);
            return Node(l.hd,
                fromList(t.fst),
                fromList(t.snd)); }
    }
}
\end{verbatim}
\end{frame}

% TODO: technical details

\end{document}
